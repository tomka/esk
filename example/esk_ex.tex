\documentclass[12pt]{article}

\usepackage{subfig}
\usepackage{tikz}
\usepackage{esk}


\title{A simple \LaTeXe{} ESK example}

\author{Tom Kazimiers}
\date{\today}


% the SubFloate macro is taken from the manual
% of Steven Douglas Cochrans subfig package.
% ftp://ctan.tug.org/tex-archive/macros/latex/contrib/subfig/subfig.pdf
\makeatletter
\newbox\sf@box
\newenvironment{SubFloat}[2][]%
{\def\sf@one{#1}%
	\def\sf@two{#2}%
		\setbox\sf@box\hbox
		\bgroup}%
{ \egroup
	\ifx\@empty\sf@two\@empty\relax
		\def\sf@two{\@empty}
	\fi
	\ifx\@empty\sf@one\@empty\relax
		\subfloat[\sf@two]{\box\sf@box}%
	\else
		\subfloat[\sf@one][\sf@two]{\box\sf@box}%
	\fi}
\makeatother


\begin{document}
\maketitle

	\tracingmacros=1
	\begin{eskdef}{general}
		def O (0,0,0) % origin
		def J [0,1,0] % rotation axis
		def dx 2.3
		def dy 2.5
		def dz dx
	\end{eskdef}
	
	\begin{eskdef}{axes}
		def axes {
		    % draw the axes
		    def ax (dx,0,0)
		    def ay (0,dy,0)
		    def az (0,0,dz)
		    line[arrows=<->,line width=.4pt](ax)(O)(ay)
		    line[arrows=->,line width=.4pt](O)(az)
		    % annote axes
		    special |\path #1 node[left] {$z$}
				   #2 node[below] {$x$}
				   #3 node[above] {$y$};|(az)(ax)(ay)
		}
	\end{eskdef}

	\begin{eskdef}{pyramid}
		def pyramid {
		    def p0 (0,2)
		    def p1 (1.5,0)
		    def N 4
		    def seg_rot rotate(360 / N, [J])
		    % draw the pyramid by rotating a line about the J axis
		    sweep[cull=false,fill=blue!20]
		      { N, [[seg_rot]] } line(p0)(p1)
		}
	\end{eskdef}

	Some examples from http://www.fauskes.net/nb/introduction-to-sketch/	
	\begin{center}
		\begin{esk}{}
			def scene{
			    polygon[fill=red!20,draw=red]
				(0,0,1)(1,0,0)(0,1,0)
			    put {rotate(25)} {polygon[fill=blue!20,draw=blue]
				(0,0,1)(1,0,0)(0,1,0)}
			}

			put { scale(2) then view((5,4,8)) } {scene}
		\end{esk}
		
		%% second row
		\begin{figure}[ht]
			\centering
			\begin{SubFloat}{\label{fig:verbone}A coordinate system.}
				\begin{minipage}[t]{1.5in}
					\centering
					\begin{esk}{general,axes}
						def scene {
							{axes}
						}
						put { view((10,4,2)) } {scene}
					\end{esk}
				\end{minipage}%
			\end{SubFloat}%
			\begin{SubFloat}{\label{fig:verbtwo}A tetrahydron.}
				\begin{minipage}[t]{1.5in}
					\centering
					\begin{esk}{general,pyramid}
						def scene {
							{pyramid}
						}
						put { view((10,4,2)) } {scene}
					\end{esk}
				\end{minipage}%
			\end{SubFloat}%
			\begin{SubFloat}{\label{fig:verbthree}Both combined.}
				\begin{minipage}[t]{1.5in}
					\centering
					\begin{esk}{general,axes,pyramid}
						def scene {
							{axes}
							{pyramid}
						}
						put { view((10,4,2)) } {scene}
					\end{esk}
				\end{minipage}%
			\end{SubFloat}%
			\label{fig:verbatim}
			\caption{Parts of a scene. In \subref{fig:verbthree} the same esk-definitions were used as in \subref{fig:verbone} and \subref{fig:verbtwo}}
		\end{figure}
		
	\end{center}
	\begin{center}
		\begin{esk}
			% truncated cone diagram
			% Modified for PGF/TikZ
			def O (0,0,0)
			def I [1,0,0]
			def J [0,1,0]
			def K [0,0,1]

			def p0 (1,2)
			def p1 (1.5,0)
			def N 8
			def seg_rot rotate(360 / N, [J])
			def dx
			  <labeled> 2
			  <> 2.3
			def dy 
			  <labeled> 2
			  <> 3.3
			def dz dx

			def basic_cone {

			  % draw the cone; this is the easy part!
			  sweep[cull=false,fill=blue!20,style=nearly transparent] { N, [[seg_rot]] } line(p0)(p1)

			  % draw the axes
			  def ax (dx,0,0)
			  def ay (0,dy,0)
			  def az (0,0,dz)
			  line[arrows=<->,line width=.4pt](ax)(O)(ay)
			  line[arrows=->,line width=.4pt](O)(az)
			  % repeat dotted as an overlay to hint at the hidden lines
			  %line[lay=over,linestyle=dotted,linewidth=.4pt](ax)(O)(ay)
			 % line[lay=over,linestyle=dotted,linewidth=.4pt](O)(az)
			  % label
			  special|\path #1 node[below] {$x$}
			     #2 node[above] {$y$}
			     #3 node[left] {$z$};|
			    (ax)(ay)(az)

			  % height measurement mark takes too much code!
			  def c0 (p0) then scale([J])
			  def hdim_ref unit((p1) - (O)) then [[seg_rot]]^2
			  def h00 (c0) + 1.1 * [hdim_ref]
			  def h01 (c0) + 1.9 * [hdim_ref]
			  def h02 (c0) + 1.8 * [hdim_ref]
			  line(h00)(h01)
			  def h10 (O) + 1.6 * [hdim_ref]
			  def h11 (O) + 1.9 * [hdim_ref]
			  def h12 (O) + 1.8 * [hdim_ref]
			  line(h10)(h11)
			  line[arrows=<->](h02)(h12)
			  def hm2 ((h02)-(O)+(h12)-(O)) / 2 + (O)
			  special|\node[ann] at #1 {$h$};|(hm2)

			  % radius measurement marks
			  def gap [0,.2,0]
			  % first r1
			  def up1 [0,3.1,0]
			  def r1 ((p1) then [[seg_rot]]^-2) + [up1]
			  def r1c (r1) then scale([J])
			  def r1t (r1) + [gap]
			  def r1b ((r1t) then scale([1,0,1])) + [gap]
			  line[arrows=<->](r1c)(r1)
			  line(r1b)(r1t)
			  def r1m ((r1) - (O) + (r1c) - (O)) / 2 + (O)
			  special |\node[ann] at #1 {$r_1$};|(r1m)
			  % same drill for r0, but must project down first
			  def up0 [0,2.7,0]
			  def r0 ((p0) then scale([1,0,1]) then [[seg_rot]]^-2) + [up0]
			  def r0c (r0) then scale([J])
			  def r0t (r0) + [gap]
			  def r0b ((p0) then [[seg_rot]]^-2) + [gap]
			  line[arrows=<->](r0c)(r0)
			  line(r0b)(r0t)
			  def r0m ((r0) - (O) + (r0c) - (O)) / 2 + (O)
			  special |\node[ann] at #1 {$r_0$};|(r0m)
			}

			def labeled_cone {
				
			  % the "ghost" of the entire cone
			  sweep[style=ghost,cull=false] { N-1, [[seg_rot]] } 
			    line(p0)(p1)

			  % for the highlighted face, we need explicit points
			  def p00 (p0) then [[seg_rot]]^-1
			  def p10 (p1) then [[seg_rot]]^-1
			  def p01 (p0)
			  def p11 (p1)
			  %polygon[showpoints=true](p00)(p10)(p11)(p01)
			  polygon[fill=red,style=semi transparent](p00)(p10)(p11)(p01)
			  % TikZ does not have a showpoints option. Do it usings specials instead
			  special|\fill[black,font=\footnotesize]
					#1 circle (1.5pt) node [above] {$P_{00}$}
					#2 circle (1.5pt) node [below] {$P_{10}$}
					#3 circle (1.5pt) node [above] {$P_{01}$}
					#4 circle (1.5pt) node [below] {$P_{11}$};|
			    (p00)(p10)(p01)(p11)
			  def mid ((p00)-(O)+(p10)-(O)+(p11)-(O)+(p01)-(O))/4+(O)
			  % The TikZ arc operation starts at the current point. We therefore
			  % need to shift it to get mid to be the center of the arc
			  special|\draw #1+(-60:.25) [yscale=1.3,->] arc(-60:240:.25);|
			    [lay=over](mid)
			  def mid_left ((p00)-(O)+(p10)-(O))/2+(O)
			  def mid_right ((p11)-(O)+(p01)-(O))/2+(O)
			  special|\path[font=\footnotesize] 
				  #1 node[left] {$j$}
				  #2 node[right] {$j\hbox{$+$}1$};|
			    (mid_left)(mid_right)
			  def top_lbl (p01) then [[seg_rot]]^2
			  def bot_lbl (p11) then [[seg_rot]]^2
			  special|\path[font=\footnotesize]
				  #1 node[right] {$i\hbox{$=$}0$}
				  #2 node[right] {$i\hbox{$=$}1$};|
			    (top_lbl)(bot_lbl)
			}




			def cone 
			  <labeled> {labeled_cone}
			  <>        {basic_cone}

			put { view((10,4,2)) } {cone}

			% Define a few styles for styling the drawing. 
			% Note the use of lay=under to force Sketch to output the code at the
			% start of the drawing.
			special |\tikzstyle{ann} = [fill=white,font=\footnotesize,inner sep=1pt]|[lay=under]
			special |\tikzstyle{ghost} = [draw=lightgray]|[lay=under]
			special |\tikzstyle{transparent cone} = [fill=blue!20,fill opacity=0.8]|[lay=under]
		\end{esk}
	\end{center}
\end{document}
  
